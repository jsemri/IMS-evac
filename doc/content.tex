\section{Introduction}
% XXX use this citation
some content~\cite{CaEvac}

\section{Problem Description}
% XXX

\section{Modeling}
In this section we describe our model based on CA (Cellular Automata). We
assume that the reader is familiar with the basic concepts of CA.

\subsection{Cells}
We work with two-dimensional space, where each cell represents a $0.4 \times
0.4$ meters square. In our model we use \emph{Moore
Neighbourhood}~\cite{Moore}, where eight cells surround a given cell. We define
following set of states, which cells can get. The set of possible states is
following $$Q = \{W,O,P,S,E,EX,PIA,PIS,OIS,PAE\}$$ where the meaning of symbols
is
\begin{itemize}
    \item W\,--\,Wall
    \item O\,--\,Obstacle
    \item P\,--\,Person
    \item S\,--\,Smoke
    \item E\,--\,Empty cell
    \item EX\,--\,Exit
    \item PIA\,--\,Person initial appearance
    \item PIS\,--\,Person in smoke
    \item OIS\,--\,Obstacle in smoke
    \item PAE\,--\,Person at the exit
\end{itemize}

\subsection{Initial Configuration}
\subsubsection{Building}
The initial configuration of walls, obstacles and exits is defined by input
pixmap. The correctness  of the input is not verified, it is left on a user.
\subsubsection{People}
The location of the people at the beginning of the simulation is by randomly
distribution over empty cells. Cells with state $E$ and $PIA$ are considered as
initial places where people can appear. However, $PIA$ cells are occupied with
higher priority. Actually this is the only difference between these cell types.
\subsubsection{Smoke}
The initial location of smoke is also randomly distributed over empty cells.
However, here the priority does not have any effect.


\subsection{Evolution Rules}
\begin{enumerate}
    \item Cells with type $W$ does not change throughout the whole simulation.
    \item Smoke can be propagated to the following cells: $O, P, E$ and $PIA$.
    Cells which can propagate smoke are $S,PIA$ and $PIS$. The cell without
    smoke can be affected with smoke with probability proportional to the
    number of smoked cells surrounding the given cell. This is defined by
    following equation $\frac{N}{8} \cdot C_{SS}$, where $N$ is the number of
    neighbouring cells affected with smoke and $C_{SS}$ is smoke spreading
    coefficient. Since the ceiling of our modeled buildings is quite high and
    we assume that people do not leave building immediately when fire
    starts, our empirical choice of of the coefficient $C_{SS}$ is 0.25.
    When a cell is being affected with the smoke its state is changes with
    respect to the following rules:
    \begin{itemize}
        \item $E \rightarrow S$, empty cell becomes smoke cell
        \item $PIA \rightarrow S$, person appearance cell becomes smoke cell
        \item $O \rightarrow OS$, obstacle becomes obstacle with smoke
        \item $P \rightarrow PIS$, person becomes person in smoke
    \end{itemize}
    \item The people, cells $P$ and $PIS$, can move to any empty cell in their
    \emph{Moore Neighbourhood}. The allowed empty cells are $E,PIA,S$ and $EX$.
    If a person reaches an exit, person leaves a system in the next evolution
    step. When person moves into specific cell, it changes its state with
    respect to the following rules.
    \begin{itemize}
        \item $E \rightarrow P$, person moves into an empty cell
        \item $PIA \rightarrow P$, person moves into appearance cell
        \item $S \rightarrow PIS$, person moves into smoke and is affected by
        smoke
        \item $EX \rightarrow PAE$, person moves into exit, and he is ready to
        leave a system, in the next evolution step this cell becomes exit
        again, so it can be occupied by another person
    \end{itemize}
    The person's previous position is recovered by its former cell type. The
    way how the pedestrians moves and behaves is desribed in the next section.
\end{enumerate}

\subsection{Pedestrains Motion}

\subsubsection{Computing the Distances}

\subsubsection{Colisions}

\section{Implementation}

\subsection{Cellular Automata}

\subsection{Image Processing}
% XXX

\section{Experiments}

\section{Conclusion}
