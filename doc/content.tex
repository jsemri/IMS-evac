\section{Introduction}
Education buildings are common examples of commercial buildings that involve a
meeting of a large number of people withing a closed area.
While designing such types of buildings, a common goal is to maximize the
productivity of the available space, but it is essential to consider the
safety aspects as well.
One of such aspects is how efficiently can emergency evacuations be carried out
due to the threat of fire.
Whether a building layout is suitable for emergency evacuation is a difficult
and rather ambiguous decision due to the fact that numerous factors must be
taken into account, such as fire and smoke spreading, behaviour of every single
individual as well as crowd dynamics and other random events.
Modelling these situations can give an insight into the safety aspects of a
building and identify its flaws.

This project addresses a question of how such evacuation processes can be
modelled using cellular automata [ims-315].
Our work is mainly based on a research by Tissera et al.~\cite{Tissera1};
several modifications enhancing pedestrian behaviour are introduced, some of
them inspired by the research by Yuan et al.~\cite{Yuan}.
We will use our approach to model evacuation processes inside two selected
buildings, namely, D and E wings of FIT BUT building~\cite{FIT}.
Using these models we will simulate different fire scenarios and collect
various statistics in order to evaluate key safety aspects of the
systems [ims-7] under investigation.

{TODO Validation TODO}

Section 2 describes some basic concepts: a theory behind cellular automata,
the laws of fire and smoke spreading, the principles behind people movement and consequences of smoke poisoning.
In Section 3 we invoke these concepts and introduce an abstract model [ims-9]
of a system.

{In Section 4 TODO}

{In Section 5 TODO}

\section{Preliminaries}
\subsection{Cellular Automata}
Let us first briefly describe a mathematical apparatus in the core of our
models, namely, cellular automata (CA)~\cite{Wolfram}.
The CA are mathematical systems with discrete values in space, time and
state.
With respect to the structure a CA can be considered as a grid of locally
interconnected cells that behave like finite automata.
The input of each finite automaton (cell) is considered to be a state of the
neighbouring cells.
During each discrete step, a cell evaluates its input and produces an
output, modifying its own state.
Therefore, the state of a cell of a cellular automaton in a particular time
step only depends on the states of its neighbouring cells and the state the
cell had in the previous time step.

The spatial framework of the CA can be specified in any number of dimensions
where cells might be of any regular shape.
Similarly, the neighbourhood of the cell can be defined in various ways.
Having established the structure of the automaton and the shape of the
neighbourhood one needs to define the set of states of a cell and the rules
that dictate transitions between these states.
In our case, we need to study the laws of fire/smoke propagation and people
motion, which is the main interest of the following subsections.

\subsection{Fire and Smoke Spreading}
The phenomenon of fire and smoke spreading is extremely complex due to the
involvement of numerous non-trivial chemical reactions and physical processes
~\cite{Ying, Curiac}.
Simulating such processes using means of CA usually involves two interconnected
automata: one for fire spreading simulation and one for smoke spreading
simulation~\cite{Curiac}.
The first one employs two factors: combustion materials that are placed in the
two adjacent cells under investigation and information about airflow velocities.
Capturing data to quantify the latter factor is not a trivial task and is
usually performed by establishing a sensor network to measure these velocities
real-time.
This is impossible for buildings that are being planned, so, again, simulations
of the airflow are used, which is a non-trivial procedure on its
own~\cite{Airflow}.
Therefore, this complex process is very often~\cite{Tissera1, Tissera2}
approximated by considering only flammability/combustibility of materials.
The model is then reduced to a simple diffusion process where a probability of
cell contamination at the current time is proportional to the contamination
level of its neighbourhood.

The CA for smoke spreading simulation are quite similar to those developed for
fire spreading with some adjustment of parameters~\cite{Curiac}.
The reason for this is that propagation speed of a smoke is typically much higher
than the one related to fire and therefore the smoke automaton evolves much
quicker than the fire automaton.
Regarding this, our preliminary prototypes showed that for buildings of
interest fire automaton remains almost stationary and, apart from generating
toxic smoke, does not influence the model whatsoever.
The reason for this lies in the fact that investigated buildings are relatively
small (area of E wing is approximately 900 m$^2$), and evacuation proceeds too
fast for the fire to spread; similar behaviour can be inspected in analogous
experiments~\cite{Tissera1}.
Exception would be the case when large areas of a building instantaniously catch
fire, although this requires a a presense of a significant amount of flammable 
material (e.g. library)~\cite{Tissera2}.

Does this imply that such buildings are safe?
Not at all, one should not neglect the speed with which smoke propagates.
This speed, again, largely depends on the spatial configuration of a building
and generated airflows.
For the exact same reasons as those mentioned before, smoke spreading is often
approximated by a simple diffusion model where each contaminated cell also
serves as a source of a smoke~\cite{Tissera1, Tissera2}.
Quantitatively, the propagation speed varies between 0.2 - 1.2 m/s~\cite{Smoke}.
We will stick with the smaller values to account for relatively high ceilings in
both E and D wings.

Also, note that in the initial phases of smoke spreading, the smoke tends to go
up driven by the buoyant forces, meaning that in the beginning no diffusion
in horizontal direction occurs~\cite{Curiac}.
In our model, this delay is compensated by two factors: i) fire alarm is not
triggered until the smoke reaches the ceiling and ii) it takes a distinguishable
amount of time for the individuals inside the building to evaluate a
seriousness of a fire in order to initiate extraction.

\subsection{People Motion}
Crowd behaviour that arises from the behaviour of every single individual is
arguably even more complex than the process of smoke spreading due to the fact
that the former assumes numerours psychological factors as well~\cite{Ying}.
A model by Tissera et al.~\cite{Tissera1} completely ignores such factors and a
behaviour of an individual is defined by his spatial distance: an individual
will proceed to the nearest exit no matter what.
A model by Yuan et al.~\cite{Yuan} also considers occupant density in a sense
that an individual might trade the nearest exit for the other one if the latter
is less crowded and therefore a person has more chances for a safe extraction.
Yuan's extended model also accounts for some characteristics of humans,
such as unadventurous effect (a person will try to exit through the same route
he entered), inertial effect (once a person moves towards a
certain exit, he is less likely to change his direction) or group effect
(group members will help one another in emergency).
An improved model by Tissera et al.~\cite{Tissera2} manages to account for human
factors by employing intelligent agents~\cite{AI} where each agent possesses
certain psychological, physiological and social characteristics that shape his
behaviour based on perceived (limited) information.

In our work we will stick with the Yuan's basic approach (an individual chooses
the nearest exit but might switch to the one that is less crowded) with one
slight modification: an individual is also not ignorant of the danger
of smoke poisoning and will prefer a clear path towards exit (choosing the
cells without the smoke) and might even change its destination exit when the
closest one is too toxic.
This description might sound vague, but we will formally define such behaviour
in the next section while introducing the abstract model.

\subsection{Effects of Smoke Inhalation}
Since we are not considering a spreading of fire inside a building and consider
only smoke, we must be aware of its impact. Smoke inhalation injury, either by
itself or in the presence of a burn, is now well-recognized~\cite{NCBI} to
result in severe lung-induced morbidity and mortality.
Severeness of the intoxication depends on the components of the smoke,
concentration and a temperature of combustion, but the critical parameter is
the time of the smoke exposure.
For carbon monoxide (the most common smoke component) of moderate concentration
($\geq$ 5000 ppm) 20 minutes of exposure is estimated to be enough to incapacite
an adult~\cite{CO1}, 5 minutes is estimated to be a threshold when
the smoke poisoning leads to permanent effects~\cite{NCBI, CO2}.
Impact can be even more severe if a person under exposure already possesses
cardiovascular or respiratory conditions~\cite{Inhalation}.

It is hard to predict the nature of the smoke in our modelled buildings as well
its concentration, so we will stick with the estimations of the average case
above and use a reference value of 5 minutes of smoke exposure to decide
whether an extraction can be considered successful.

\section{Model Description}
Let us first describe a simplified version of a model to familiarize the reader with 
the basic rules; later we will introduce one advanced property that completes the 
model.
We consider a finite two-dimensional grid of square cells with closed boundaries; 
each cell represents a $0.4 \times 0.4$ meters square, which is considered to be the 
space occupied by a person in a crowd with maximal
density~\cite{Density1, Density2}.
We use \emph{Moore Neighbourhood}~\cite{Moore}, which is a common choice when one 
wants to allow a cell to act along all possible directions.
A discrete time step represents 0.3 seconds of real time, which is estimated to be 
the time required for a pedestrian to move 0.4 meters
(size of a cell side)~\cite{Density1}.

\subsection{Cell States}
The set of all possible cell states $Q$ is the following
$$Q = \{W,O,P,S,E,EX,PS,OS,PE\}$$ where:
\begin{itemize}
    \item W\,--\,Wall
    \item O\,--\,Obstacle
    \item P\,--\,Person
    \item S\,--\,Smoke
    \item E\,--\,Empty cell
    \item EX\,--\,Exit
    \item PS\,--\,Person with Smoke
    \item OS\,--\,Obstacle with Smoke
    \item PE\,--\,Person at the Exit
\end{itemize}

A \emph{Wall} cell represents an external or internal wall that cannot be occupied 
by a person or be penetrated by smoke; \emph{Obstacle} cells (tables, vending 
machines etc.) also serve as a barrier for individuals, but do not stop the smoke 
spreading (hence $OS$ exists as well).
Other state names are self-explanatory.

There is one more cell state, $PI$ (Person Initial position) which is used during 
the initialization of the CA and is not involved in its evolution.
For now we assume that a specific initial configuration is given and in the next 
section we will describe how such configuration is loaded into the program.

Each cell also carries one value, $sd$ (\emph{spatial distance}), that represents a 
distance (in cell hops) to the nearest exit.

\subsection{Evolution Rules}
\begin{enumerate}
    \item Wall cell preserves its state throughout the whole simulation.

    \item A cell with smoke ($S$, $PS$ or $OS$) at time $t$ will also have smoke at 
    time $t+1$. If at time $t$ a certain cell does not have a smoke (states $E$, $P$ 
    or $O$), but some of its adjacent cells have smoke, with a certain probability 
    such cell will have smoke at time $t+1$ ($E$ will become $S$, $P$ will become
    $PS$, $O$ will become $OS$). This probability is proportional to the number of 
    cells with smoke in the neighbourhood; the exact value is computed using an 
    expression $C_{SS} \cdot \frac{N}{8}$, where $N$ is the number of adjacent cells 
    affected by smoke and $C_{SS}$ is smoke spreading coefficient. A smoke 
    propagation speed of $0.2-1.2$m/s translates to $0.15-0.9$ cells per step; 
    using assumptions stated in the previous section, we choose a value of 
    normalizing coefficient $C_{SS}$ to be 0.25. 
    
    \item In one discrete step a person (states $P$ and $PS$) can any move to any 
    cell in his neighbourhood, provided that i) the target cell is unoccupied (
    states $E$, $S$ and $EX$) and ii) the distance to the exit $sd$ of the target 
    cell is less than or equal to the one that is currently occupied. When a person 
    reaches an exit (state $EX \rightarrow PE$), he remains there until the end of 
    the discrete step (simulating the blocking of an exit) and is collected at the 
    beginning of the next step, freeing the exit. Other aspects related to the 
    people motion are described in the next subsection.
\end{enumerate}

\subsection{People Motion}
In order to drive the movement of each individual we need to compute for each cell 
that can be occupied (neither walls nor obstacles) its spatial distance $sd$ to the closest exit.
We consider the cellular space as a graph, where each cell represents a node and all 
the edges connecting adjacent cells have weight one.
Having multiple exits, a modified version of the Dijkstra's shortest path algorithm 
is applied where the edges are reversed and exits (destinations) are treated as 
sources (multiple source to single destination, MSSD)~\cite{Dijkstra}.

As was mentioned before, a person will try to move closer to the exit by selecting 
those adjacent cells that have an $sd$ value less than the one that is currently 
occupied.
In case when there is more than one such cell, a cell is selected at random.
To avoid collisions (when two or more individuals aim to occupy the same cell during 
transition) we perform update of person cells sequentially.
The order of update is random and is different during every step in order to avoid 
any bias.

In case a person cannot move to the cell nearer to exit (because it is being 
occupied by another person), with certain probability he can move to the cell with 
the same spatial distance.
This probability is denoted as \emph{chaosTODO} and has a value of TODO.
This approach solves an issue addressed by Tissera et al.~\cite{Tissera1} when 
blocked individuals were standing still which sometimes leads to unrealistic 
behaviour.
This probabilistic shift can also simulate by-passing (on a local scope) of blockers 
ahead.

Finally, we have to answer a question how to enhance the abilities of individuals to 
evaluate current situation so that they are able to recognize bad routes and bad 
exits, either overcrowded or filled with smoke.
This is achieved by introducing a notion of perceived distances.

\subsection{Perceived Distances}
In order to explain the idea behind perceived distances we will first modify our 
existing model and then justify these changes when the goal becomes more apparent. 
First, we change $sd$ property of a cell to $pd$, which stands for \emph{perceived 
distance}.
Next, for MSSD path evaluation we use the following modifications:

\begin{itemize}
    \item a distance from any cell to the cell that is occupied by an individual (state $P$) is $C_{op}$
    \item a distance from any cell to the cell that is filled with smoke (state $S$) is $C_{sp}$
    \item a distance from any cell to the cell that is both occupied by an individual and filled with smoke (state $PS$) is $C_{op} \cdot C_{sp}$
    \item all other edges have weight $1$ (as before)
\end{itemize}
for $C_{op} \geq 1, C_{sp} \geq 1$.
A person then selects a cell which has a minimum value of $pd$.

In other words, when a target cell is occupied or filled with smoke, it is \emph{
perceived} as being further from the current one.
Using these modifiers for a solution of MSSD problem allows for the heavily 
overcrowded routes to magnify their $pd$ property.
Since the configuration of automaton is not stationary and the weights of the edges 
dynamically change, perceived distances must be recomputed in the beginning of each 
step.
As a consequence of this, a person is now able (in each step) to evaluate the 
current situation, identify overcrowded or intoxicated areas and try to avoid them. 
Please note that the presence of smoke or a person several steps ahead will not slow 
down an individual: the spatial distance between two adjacent cells is still 0.4 
meters and it still takes 0.3 seconds to cover this distance.
What changes is the \emph{perception} (hence the name) of the surrounding area that
influences a behavior of an individual.

We still have to assign concrete values to parameters $C_{op}$ and $C_{sp}$.
Note that for $C_{op} = C_{sp} = 1$ the perceived distance is reduced to the spatial 
distance and individuals act solely upon information about nearest exits. 
Unfortunately, we failed to find any statistical data about a human perception of 
the crowds or the smoke that would suggest the choice of the parameters, neither 
could we derive them based upon real life experiments, for obvious reasons. 
Empirical value of 5.0 for $C_{sp}$ was selected based on experiments with the 
model; such value leads to behaviour when a person would trade an exit at a
(spatial) distance 4.0 in favor of the exit at the distance 5.0 in order to avoid 
the smoke; conversely, a person is not afraid to cross 0.4 meters of smoked area 
towards exit when the next closest one is ten meters away.
The exact same argument applies to $C_{op}$ for which a value of 2.0 was chosen.
Yuan's model ~\cite{Yuan} also quantifies this perception by means of a set
of parameters.

\section{Implementation}
We divide the handling of input and output from application logic. The
input and output is based on images processing. The application core is briefly
described in the next section.

\subsection{Cellular Automata}
We represent CA with class \texttt{CA} under namespace \texttt{Evacuation}.
Implementation of class \texttt{CA} is quite simple. \texttt{CA} provides
methods for constructing object from file, computing statistics, outputting an
image and other useful features. The main function called \texttt{evolve},
which returns \texttt{true} if there are still some people to evacuate, changes
state of the automaton. Basically the whole simulation consist of few lines of
code.

\lstset{language=C++,
basicstyle=\ttfamily,
keywordstyle=\color{blue}\ttfamily,
stringstyle=\color{red}\ttfamily,
commentstyle=\color{magenta}\ttfamily,
morecomment=[l][\color{magenta}]{\#}
}
\bigskip
\begin{lstlisting}
while (ca.evolve() == true)      // evolve until all people are evacuated
{
    usleep(simulation_time);     // wait for a while
    ca.show();                   // show current state of the automaton
}
\end{lstlisting}

\subsection{Initial Configuration}
The initial configuration of walls, obstacles and exits is defined by input
pixmap. The correctness  of the input is not verified, it is left on a user.
The initial location of people and smoke is solved by program and it is
described in next sections.

\subsubsection{People}
The location of the people at the beginning of the simulation is by randomly
distribution over empty cells. Cells with state $E$ and $PI$ are considered as
initial places where people can appear. However, $PI$ cells are occupied with
higher priority. Actually this is the only difference between these cell types.

\subsubsection{Smoke}
The initial location of smoke is also randomly distributed over empty cells.
However, here the priority does not have any effect.

% XXX

\section{Experiments}

\section{Conclusion}