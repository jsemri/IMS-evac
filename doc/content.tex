\section{Introduction}
% XXX use this citation
some content~\cite{CaEvac}

\section{Problem Description}
% XXX

\section{Modeling}
In this section we describe our model based on CA (Cellular Automata). We
assume that the reader is familiar with the basic concepts of CA.

\subsection{Cells}
We work with two-dimensional space, where each cell represents a $0.4 \times
0.4$ meters square. In our model we use \emph{Moore
Neighbourhood}~\cite{Moore}, where eight cells surround a given cell.
The set of possible states $Q$ which each cell can get is following
$$Q = \{W,O,P,S,E,EX,PIA,PIS,OIS,PAE\}$$ where the meaning of symbols
is
\begin{itemize}
    \item W\,--\,Wall
    \item O\,--\,Obstacle
    \item P\,--\,Person
    \item S\,--\,Smoke
    \item E\,--\,Empty cell
    \item EX\,--\,Exit
    \item PIA\,--\,Person Initial Appearance
    \item PIS\,--\,Person In Smoke
    \item OIS\,--\,Obstacle In Smoke
    \item PAE\,--\,Person At the Exit
\end{itemize}

\subsection{Initial Configuration}
The initial configuration of walls, obstacles and exits is defined by input
pixmap. The correctness  of the input is not verified, it is left on a user.
The initial location of people and smoke is solved by program and it is
described in next sections.
\subsubsection{People}
The location of the people at the beginning of the simulation is by randomly
distribution over empty cells. Cells with state $E$ and $PIA$ are considered as
initial places where people can appear. However, $PIA$ cells are occupied with
higher priority. Actually this is the only difference between these cell types.
\subsubsection{Smoke}
The initial location of smoke is also randomly distributed over empty cells.
However, here the priority does not have any effect.


\subsection{Evolution Rules}
\begin{enumerate}
    \item Cells with type $W$ does not change throughout the whole simulation.
    \item Smoke can be propagated to the following cells: $O, P, E$ and $PIA$.
    Cells which can propagate smoke are $S,PIA$ and $PIS$. The cell without
    smoke can be affected with smoke with probability proportional to the
    number of smoked cells surrounding the given cell. This is defined by
    following equation $\frac{N}{8} \cdot C_{SS}$, where $N$ is the number of
    neighbouring cells affected with smoke and $C_{SS}$ is smoke spreading
    coefficient. Since the ceiling of our modeled buildings is quite high and
    we assume that people do not leave building immediately when fire
    starts, our empirical choice of of the coefficient $C_{SS}$ is 0.25.
    When a cell is being affected with the smoke its state is changed with
    respect to the following rules:
    \begin{itemize}
        \item $E \rightarrow S$, an empty cell becomes smoke cell
        \item $PIA \rightarrow S$, a person appearance cell becomes a smoke cell
        \item $O \rightarrow OS$, an obstacle becomes an obstacle with smoke
        \item $P \rightarrow PIS$, a person becomes a person in smoke
    \end{itemize}
    \item The people, cells $P$ and $PIS$, can move to any empty cell in their
    \emph{Moore Neighbourhood}. The allowed empty cells are $E,PIA,S$ and $EX$.
    If a person reaches an exit, person leaves a system in the next evolution
    step. When person moves into specific cell, it changes its state with
    respect to the following rules.
    \begin{itemize}
        \item $E \rightarrow P$, a person moves into an empty cell
        \item $PIA \rightarrow P$, a person moves into an appearance cell
        \item $S \rightarrow PIS$, a person moves into a smoke and is affected
        by smoke
        \item $EX \rightarrow PAE$, a person moves into an exit, and is
        ready to leave a system, in the next evolution step this cell becomes
        exit again, so it can be occupied by another person
    \end{itemize}
    The person's previous position is recovered by its former cell type. The
    way how the pedestrians moves and behaves is described in the next section.
\end{enumerate}

\subsection{Pedestrians Motion}
Person movement is based on picking the smallest distance to the exit. However,
to avoid unrealistic motion we try to refine it with some extended features.

\subsubsection{Computing the Distances}
% XXX dijsktra, recomputing, etc.

\subsubsection{Collisions}
In case a person cannot move to the cell nearer to exit (mostly because of
another person), the person can move to the cell with same exit distance with
some probability. This probability is denoted as \emph{chaos}. It was defined
because of the fact that a person in reality does not immediately know the real
exit distance. This also simulates by-passing of blocking people and not
creating unrealistic queues before exits.

\section{Implementation}
We divide the handling of input and output from application logic. The
input and output is based on images processing. The application core is briefly
described in the next section.

\subsection{Cellular Automata}
We represent CA with class \texttt{CA} under namespace \texttt{Evacuation}.
Implementation of class \texttt{CA} is quite simple. \texttt{CA} provides
methods for constructing object from file, computing statistics, outputting an
image and other useful features. The main function called \texttt{evolve},
which returns \texttt{true} if there are still some people to evacuate, changes
state of the automaton. Basically the whole simulation consist of few lines of
code.

\lstset{language=C++,
basicstyle=\ttfamily,
keywordstyle=\color{blue}\ttfamily,
stringstyle=\color{red}\ttfamily,
commentstyle=\color{magenta}\ttfamily,
morecomment=[l][\color{magenta}]{\#}
}
\bigskip
\begin{lstlisting}
while (ca.evolve() == true)      // evolve until all people are evacuated
{
    usleep(simulation_time);     // wait for a while
    ca.show();                   // show current state of the automaton
}
\end{lstlisting}

\subsection{Image Processing}
% XXX

\section{Experiments}

\section{Conclusion}
