\section{Introduction}
Education buildings are common examples of commercial buildings that involve a
meeting of a large number of people withing a closed area.
While designing such types of buildings, a common goal is to maximize the
productivity of the available space, but it is essential to consider the
safety aspects as well.
One of such aspects is how efficiently can emergency evacuations be carried out
due to the threat of fire.
Whether a building layout is suitable for emergency evacuation is a difficult
and somewhat ambiguous decision due to the fact that numerous factors must be
taken into account, such as fire and smoke spreading, behaviour of every single
individual as well as crowd dynamics and other random events.
Modelling these situations can give an insight into the safety aspects of a
building and identify its flaws.

This project addresses a question of how such evacuation processes can be
modelled using cellular automata [ims-315].
Our work is mainly based on a research by Tissera et al.~\cite{Tissera1};
several modifications enhancing pedestrian behaviour are introduced, some of
them inspired by the research by Yuan et al.~\cite{Yuan}.
We will use our approach to model evacuation processes inside two selected
buildings, namely, D and E wings of FIT BUT building~\cite{FIT}.
Using these models we will simulate different fire scenarios and collect
various statistics in order to evaluate key safety aspects of the
systems [ims-7] under investigation.

{TODO Validation TODO}

Section 2 describes some basic concepts: a theory behind cellular automata,
the laws of fire and smoke spreading and the principles behind people movement.
Also, in order to evaluate a level of emergency safety of the buildings of
interest, we will try to measure an impact of the fire on each individual
inside, which is an interest of subsecion 2.4.
In Section 3 we invoke these concepts and introduce an abstract model [ims-9]
of a system.

{In Section 4 TODO}

{In Section 5 TODO}

\section{Preliminaries}
\subsection{Cellular Automata}
Let us first briefly describe a mathematical apparatus in the core of our
models, namely, cellular automata (CA)~\cite{Wolfram}.
The CA are mathematical systems with discrete values in space, time and
state.
With respect to the structure a CA can be considered as a grid of locally
interconnected cells that behave like finite automata.
The input of each finite automaton (cell) is considered to be a state of the
neighbouring cells.
During each discrete step, a cell evaluates its input and produces an
output, modifying its own state.
Therefore, the state of a cell of a cellular automaton in a particular time
step only depends of the states of its neighbouring cells and the state the
cell had in the previous time step.

The spatial framework of the CA can be specified in any number of dimensions
where cells might be of any regular shape.
Similarly, the neighbourhood of the cell can be defined in various ways.
Having established the structure of the automaton and the shape of the
neighbourhood one needs to define the set of states of a cell and the rules
that dictate transitions between the states.
In our case, we need to study the laws of fire/smoke spreading and people
motion, which is the main interest of the following subsections.

\subsection{Fire and smoke spreading}
The phenomenon of fire and smoke spreading is extremely complex due to the
involvement of numerous non-trivial chemical reactions and physical processes
~\cite{Ying}~\cite{Curiac}.
Simulating such processes using means of CA usually involves two interconnected
automata: one for fire spreading simulation and one for smoke spreading
simulation~\cite{Curiac}.
The first one employs two factors: combustion materials that are placed in the
two adjacent cells under investigation and information about airflow velocities.
Capturing data to quantify the latter factor is not a trivial task and is
usually performed by establishing a sensor network to measure these velocities
real-time.
This is impossible for buildings that are being planned, so, again, simulations
of the airflow are used, which is a non-trivial procedure on its
own~\cite{Airflow}.
Therefore, this complex process is very often~\cite{Tissera1}~\cite{Tissera2}
approximated by considering only flammability/combustibility of materials.
The model is then reduced to a simple diffusion process where a probability of
cell contamination at the current time is proportional to the contamination
level of its neighbourhood.

The CA for smoke spreading simulation are quite similar to those developed for
fire spreading with some adjustment of parameters~\cite{Curiac}.
The reason for this is that propagation speed of a smoke is usually much higher
than the one related to fire and therefore the smoke automaton evolves much
quicker than the fire automaton.
Regarding this, our preliminary prototypes showed that for buildings of
interest fire automaton remains almost stationary and, apart from generating
toxic smoke, does not influence the model whatsoever.
The reason for this lies in the fact that investigated buildings are relatively
small (area of E wing is approximately 900 m$^2$), and evacuation proceeds too
fast for the fire to spread (ignoring the case when the whole complex
completely randomly and instantaniously catches fire).

Does this imply that such buildings are safe?
Not at all, one should not neglect the speed with which smoke propagates.
This speed, again, largely depends on the spatial configuration of a building
and generated airflows.
For the exact same reasons as those mentioned before, smoke spreading is often
approximated by a simple diffusion model where each contaminated cell also
serves as a source of a smoke~\cite{Tissera1}~\cite{Tissera2}.
Quantitatively, the propagation speed varies between 0.2 - 1.2 m/s~\cite{Smoke}.
We will stick to the smaller values to account for relatively high ceilings in
both E and D wings.

Also, note that in the initial phases of smoke spreading, the smoke tends to go
up driven by the buoyant forces, meaining that in the beginning no diffusion
in horizontal direction occurs~\cite{Curiac}.
In our model, this delay is compensated by two factors: i) fire alarm is not
triggered until the smoke reaches the ceiling and ii) it takes a distinguishable
amount of time for the individuals inside the building to evaluate a
seriousness of a fire in order to initiate extraction.

\subsection{People motion}
Crowd behaviour that arises from the behaviour of every single individual is
arguably even more complex than the process of smoke spreading due to the fact
that the former assumes numerours psychological factors as well~\cite{Ying}.
A model by Tissara et al.~\cite{Tissera1} completely ignores such factors and a
behaviour of an individual is defined by his spatial distance: an individual
will proceed to the nearest exit no matter what.
A model by Yuan et al.~\cite{Yuan} also considers occupant density in a sense
that an individual might trade the nearest exit for the other one if the latter
is less crowded and therefore a person has more chances for a safe extraction.
Yuan's extended model also accounts for some characteristics of humans,
such as unadventurous effect (a person will try to exit through the same route
he entered a building because occupants generally do not want to use a new exit
they have no experience with), inertial effect (once a person moves towards a
certain exit, he is unlikely to change his direction) or group effect(group
members will help one another in emergency).
An improved model by Tissara et al.~\cite{Tissera2} manages to account for human
factors by employing intelligent agents~\cite{AI} where each agent possesses
certain psychological, psysiological and social characteristics that shape his
behaviour based on perceived (limited) information.

In our work we will stick to the Yuan's basic approach (an individual chooses
the nearest exit but might switch to the one that is less crowded) with one
slight modification: an individual is also not ignorant of the danger
of smoke poisoning and will prefer a clear path towards exit (choosing the
cells without the smoke) and might even change its destination exit when the
closest one is too toxic.
This description might sound vague, but we will formally define such behaviour
in the next section while introducing the abstract model.

\subsection{Effects of smoke inhalation}
Since we are not considering a spreading of fire inside a building and consider
only smoke, we must be aware of its impact. Smoke inhalation injury, either by
itself or in the presence of a burn, is now well-recognized~\cite{NCBI} to
result in severe lung-induced morbidity and mortality.
Severeness of the intoxication depends on the components of the smoke,
concentration and a temperature of combustion, but the critical parameter is
the time of the smoke exposure.
For carbon monoxide (the most common smoke component) of moderate concentration
($\geq$ 5000 ppm) 20 minutes of exposure is estimated to be enough to incapacite
an adult~\cite{CO1}, 5 minutes is estimated to be a threshold when
the smoke poisoning leads to permanent effects~\cite{NCBI}~\cite{CO2}.
Impact can be even more severe if a person under exposure already possesses
cardiovascular or respiratory conditions~\cite{Inhalation}.

It is hard to predict the nature of the smoke in our modelled buildings as well
its concentration, so we will stick to the estimations of the average case
above and use a reference value of 5 minutes of smoke exposure to evaluate
whether an extraction can be considered successful.

\section{Modeling}
%We will consider a finite two-dimensional grid of square cells, we will stick to the Moore-shape~\cite{Moore} neighbourhood (eight surrounging cells), which is a common choice when one wants to allow a cell to act along all possible directions.

In this section we describe our model based on CA (Cellular Automata). We
assume that the reader is familiar with the basic concepts of CA.

\subsection{Cells}
We work with two-dimensional space, where each cell represents a $0.4 \times
0.4$ meters square. In our model we use \emph{Moore
Neighbourhood}~\cite{Moore}, where eight cells surround a given cell.
The set of possible states $Q$ which each cell can get is following
$$Q = \{W,O,P,S,E,EX,PIA,PIS,OIS,PAE\}$$ where the meaning of symbols
is
\begin{itemize}
    \item W\,--\,Wall
    \item O\,--\,Obstacle
    \item P\,--\,Person
    \item S\,--\,Smoke
    \item E\,--\,Empty cell
    \item EX\,--\,Exit
    \item PIA\,--\,Person Initial Appearance
    \item PIS\,--\,Person In Smoke
    \item OIS\,--\,Obstacle In Smoke
    \item PAE\,--\,Person At the Exit
\end{itemize}

\subsection{Initial Configuration}
The initial configuration of walls, obstacles and exits is defined by input
pixmap. The correctness  of the input is not verified, it is left on a user.
The initial location of people and smoke is solved by program and it is
described in next sections.
\subsubsection{People}
The location of the people at the beginning of the simulation is by randomly
distribution over empty cells. Cells with state $E$ and $PIA$ are considered as
initial places where people can appear. However, $PIA$ cells are occupied with
higher priority. Actually this is the only difference between these cell types.
\subsubsection{Smoke}
The initial location of smoke is also randomly distributed over empty cells.
However, here the priority does not have any effect.


\subsection{Evolution Rules}
\begin{enumerate}
    \item Cells with type $W$ does not change throughout the whole simulation.
    \item Smoke can be propagated to the following cells: $O, P, E$ and $PIA$.
    Cells which can propagate smoke are $S,PIA$ and $PIS$. The cell without
    smoke can be affected with smoke with probability proportional to the
    number of smoked cells surrounding the given cell. This is defined by
    following equation $\frac{N}{8} \cdot C_{SS}$, where $N$ is the number of
    neighbouring cells affected with smoke and $C_{SS}$ is smoke spreading
    coefficient. Since the ceiling of our modeled buildings is quite high and
    we assume that people do not leave building immediately when fire
    starts, our empirical choice of of the coefficient $C_{SS}$ is 0.25.
    When a cell is being affected with the smoke its state is changed with
    respect to the following rules:
    \begin{itemize}
        \item $E \rightarrow S$, an empty cell becomes smoke cell
        \item $PIA \rightarrow S$, a person appearance cell becomes a smoke cell
        \item $O \rightarrow OS$, an obstacle becomes an obstacle with smoke
        \item $P \rightarrow PIS$, a person becomes a person in smoke
    \end{itemize}
    \item The people, cells $P$ and $PIS$, can move to any empty cell in their
    \emph{Moore Neighbourhood}. The allowed empty cells are $E,PIA,S$ and $EX$.
    If a person reaches an exit, person leaves a system in the next evolution
    step. When person moves into specific cell, it changes its state with
    respect to the following rules.
    \begin{itemize}
        \item $E \rightarrow P$, a person moves into an empty cell
        \item $PIA \rightarrow P$, a person moves into an appearance cell
        \item $S \rightarrow PIS$, a person moves into a smoke and is affected
        by smoke
        \item $EX \rightarrow PAE$, a person moves into an exit, and is
        ready to leave a system, in the next evolution step this cell becomes
        exit again, so it can be occupied by another person
    \end{itemize}
    The person's previous position is recovered by its former cell type. The
    way how the pedestrians moves and behaves is described in the next section.
\end{enumerate}

\subsection{Pedestrians Motion}
Person movement is based on picking the smallest distance to the exit. However,
to avoid unrealistic motion we try to refine it with some extended features.

\subsubsection{Computing the Distances}
% XXX dijsktra, recomputing, etc.

\subsubsection{Collisions}
In case a person cannot move to the cell nearer to exit (mostly because of
another person), the person can move to the cell with same exit distance with
some probability. This probability is denoted as \emph{chaos}. It was defined
because of the fact that a person in reality does not immediately know the real
exit distance. This also simulates by-passing of blocking people and not
creating unrealistic queues before exits.

\section{Implementation}
We divide the handling of input and output from application logic. The
input and output is based on images processing. The application core is briefly
described in the next section.

\subsection{Cellular Automata}
We represent CA with class \texttt{CA} under namespace \texttt{Evacuation}.
Implementation of class \texttt{CA} is quite simple. \texttt{CA} provides
methods for constructing object from file, computing statistics, outputting an
image and other useful features. The main function called \texttt{evolve},
which returns \texttt{true} if there are still some people to evacuate, changes
state of the automaton. Basically the whole simulation consist of few lines of
code.

\lstset{language=C++,
basicstyle=\ttfamily,
keywordstyle=\color{blue}\ttfamily,
stringstyle=\color{red}\ttfamily,
commentstyle=\color{magenta}\ttfamily,
morecomment=[l][\color{magenta}]{\#}
}
\bigskip
\begin{lstlisting}
while (ca.evolve() == true)      // evolve until all people are evacuated
{
    usleep(simulation_time);     // wait for a while
    ca.show();                   // show current state of the automaton
}
\end{lstlisting}

\subsection{Image Processing}
% XXX

\section{Experiments}

\section{Conclusion}
